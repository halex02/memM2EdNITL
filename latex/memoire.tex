
		\documentclass[12pt, a4paper]{book}
		
		\newcommand{\styleabbr}[1]{\textit{#1}}
		
\usepackage[utf8]{inputenc} \usepackage[T1]{fontenc}		
\usepackage[series={A,B,C,D},noend,noeledsec,nofamiliar,noledgroup]{reledmac}
		
\usepackage{polyglossia} \setmainlanguage{french}
\setotherlanguage{latin}
		
\begin{document} \title{Édition critique de} \maketitle		
			\beginnumbering
			
	
	
		
			
				
			
			
\chapter{ De itinere u\styleabbr{er}sus Siriam cui\styleabbr{us}
t\styleabbr{er}ra sancta est pars. }
				
	\pstart
	
Egressus ego de Alem\styleabbr{an}nia t\styleabbr{er}ra
nat\styleabbr{iui}tatis mee et
L\styleabbr{um}bardi\styleabbr{am} t\styleabbr{ra}nsiens ad
litt\styleabbr{us} Medit\styleabbr{er}ranei maris
\styleabbr{pro}pe ciuitat\styleabbr{em}, \styleabbr{que}
Naul\styleabbr{um} d\styleabbr{icitur}, in ripis Ian[u]e
sit\styleabbr{am} \styleabbr{pro}spere qↄmũ \styleabbr{per}ueni
et ubi gale\styleabbr{am} b\styleabbr{ene} armat\styleabbr{am}
et ad \styleabbr{omn}ia munit[a] asc\styleabbr{en}di et ad
\styleabbr{pro}se\styleabbr{que}nd\styleabbr{um} iter
ultramarinum naute remis uelo puppi \styleabbr{in}sistebant.
				
	\pend
	
				
	\pstart
	
Dicitur au\styleabbr{tem} hoc mare
Medit\styleabbr{er}rane\styleabbr{um} q\styleabbr{ui}a
p\styleabbr{ri}ncipalib\styleabbr{us} m\styleabbr{un}di
\styleabbr{par}tib\styleabbr{us} uidel\styleabbr{icet} Asie
Europe et Affrice int\styleabbr{er}iacet i\styleabbr{ps}as se
\styleabbr{et} suis braciis ab \styleabbr{in}uicem distininans
\styleabbr{habe}t en\styleabbr{im} a sept\styleabbr{en}trione
\styleabbr{et} occid\styleabbr{en}te Europ\styleabbr{am} ad
orient\styleabbr{em} Asiam et ad austr\styleabbr{am}
Affric\styleabbr{am} et uno q\styleabbr{ui}dem bracio
q\styleabbr{uo} att\styleabbr{in}git Hispani\styleabbr{am} et
strict\styleabbr{um} \styleabbr{quod} q\styleabbr{ui}dem de
Marroch uulgari\styleabbr{ter} uoca\styleabbr{tur}
c\styleabbr{on}tinua\styleabbr{tur} hoc mare Me
di\styleabbr{ter}rane\styleabbr{um} c\styleabbr{um} occeano
maiori seu c\styleabbr{in}gulo terre scil\styleabbr{icet} maximo
\styleabbr{quod} braci\styleabbr{um} s\styleabbr{an}cti Georgii
d\styleabbr{icitur} c\styleabbr{on}tinua\styleabbr{tur}
c\styleabbr{um} mari P\styleabbr{on}tico \styleabbr{quod}
q\styleabbr{ua}si null\styleabbr{am} \styleabbr{habe}t
\styleabbr{in}sul\styleabbr{m} \styleabbr{un}de \styleabbr{et}
mare mai\styleabbr{us} .
				
	\pend
	
				
	\pstart
	
In hoc gᵒ mari beat\styleabbr{us} Clemens papa submↄsus fuit ꝓpe
Cersonam et h̃itaculũ marturii in modũ tẽpli marmorei
manib\styleabbr{us} \styleabbr{}gelicis cõstructũ est. aliud
mare uersus orientẽ ultra ciuitatẽ sara qũa tenet Tartarius de
Guinama \styleabbr{et} dĩ mare Caspiũ qd nec occeano nec
Meditↄraneo nec Põtico mari aliquo apparẽti bracio copulatur.
Asserũt t̃n aliqui qⁱbusd\styleabbr{} nixi experiẽtis q s per
subterratũ gurgitẽ Põtico qd sibi ꝓpĩqui\styleabbr{us} est et ꝑ
Ↄns aliis marib9 cõtinuatↄ.
				
	\pend
	
				
	\pstart
	
Hoc braciũ s̃cti Georgii qd dixim\styleabbr{us} diuidit
Europ\styleabbr{} \styleabbr{et} Asi\styleabbr{} minorẽ q̃
ꝓfecto ut sepe t\styleabbr{}gunt maioris Asie ꝓuincia est. Hoc
aut̃ braciũ uulgaritↄ bucca Cõstantinopolitana dĩ eo q suꝓ ipsũ
in littore Europe egregia metropolis Cõstantinopol̃ sita est q̃
et noua Roma dĩ. Hec ciuitas solẽpnissima in optimo mũdi loco
t\styleabbr{} \styleabbr{ratione} aeris maris tↄre uinorũ
bonorŭq cõfluentia lorat est portũ optimũ et maxiᵐ muris
ualidissimis cĩgitur figura pacens triangulari cui\styleabbr{us}
duo latera uↄsus mare sũt et tↄtiũ uersus terr\styleabbr{}.
				
	\pend
	
				
	\pstart
	
Ĩ hac ciuitate multi sũt ecclesie et fuerũt plures ultra modũ
pulcre opere musaico marmorib\styleabbr{us} et singulari modo
cõstructure mirabilis plura q sũt palatia p̃hↄrima ĩ eadem tenet
t̃n pⁱncipatum ĩ ip̃a ciuitate eccl̃a s̃cte Sophie sapiẽtie qↄ
xp̃s est qu\styleabbr{} Iustinian\styleabbr{us} imperator
sanctissim\styleabbr{us} fũdauit et mirabl̃r
fĩgularib\styleabbr{us} pↄrogatiuis ⁊ pↄconiis decorauit aut̃ q
sub celo a cõdito orbe non fuit tale edificiũ cõpletũ in
nobilitate uidelic⁊ et magnitudine cetↄis parib\styleabbr{us}
parandum. [f237r]In ista gᵒ pↄtiosissima eccl̃a est ymago
imꝑatoris Iustiniani eques ⁊ de ere fusa imꝑiali eminens
dignitate corona deaurata maxĩe qᵃntitatis ĩsignita in manu
sinistra pomũ qd orbẽ rep̃entat cruce suꝑposita tenens
dextr\styleabbr{} q cõtra orientẽ leuans ad modũ pⁱncips minas
rebellib\styleabbr{us} apponẽtis statua suꝑ qu\styleabbr{} ymago
posita est altissima existit ex petris magnis et cernento
fortissimo glutinata.
				
	\pend
	
				
			
		
	

\endnumbering \end{document}	